\chapter{Compiling \faust Program for \ros Use}
\label{chap:compilation}

To compile a \faust program for a \ros use, you can use either the \lstinline'faust2ros' command, or the \lstinline'faust2rosgtk' one, which adds a gtk graphic user interface to the simple \lstinline'faust2ros' command.
Note that all the \faust compilation options remain.

\section{Compiling in a \faust Archive}
In order to compile a DSP file into a \faust archive, just type the command followed by your file : 
\begin{lstlisting}
faust2ros file.dsp
\end{lstlisting}
It should output :
\begin{lstlisting}
file.zip;
\end{lstlisting}
and the resulting  \lstinline'file.zip' folder should contain the following elements:

\begin{tabular}{ll}
	\lstinline'faust_msgs' 		&messages package to handle faust messages\\
	\lstinline'file'		&package containing a .cpp file corresponding to the DSP file\\
\end{tabular}

If the DSP file is not in the current directory, make sure to type the right path. For instance :
\begin{lstlisting}
faust2ros ~/faust/examples/myfile.dsp
\end{lstlisting}


\paragraph{Comment:}If you want to use the \lstinline'faust2rosgtk' command, the output will have a \lstinline'_gtk' extension. For instance :
\begin{lstlisting}
faust2rosgtk file.dsp
\end{lstlisting}
should output :
\begin{lstlisting}
file_gtk.zip;
\end{lstlisting}
and the resulting  \lstinline'file_gtk.zip' folder should contain the following elements:

\begin{tabular}{ll}
	\lstinline'faust_msgs' 		&messages package to handle faust messages\\
	\lstinline'file_gtk'		&package containing a .cpp file corresponding to the DSP file\\
\end{tabular}

\section{Compiling in a Workspace}
Thanks to the option \lstinline'-install', you have the possibility to create a package from your DSP file directly in the a workspace you chose.
Just type :
\begin{lstlisting}
faust2ros -install faust_ws file.dsp
\end{lstlisting}

It should output : 
\begin{lstlisting}
file.cpp; 
\end{lstlisting}
and you should have a faust\_ws repository looking like this :\\

\dirtree{%
.1 faust\_ws. 
	.2 build. 
	.2 devel. 
	.2 src. 
		.3 faust\_msgs : \begin{minipage}[t]{10cm}
						Messages Package \\
						Files to handle \faust messages{.}
					   \end{minipage}. 
			.4 include. 
			.4 msg. 
				.5 param\_faust.msg. 
			.4 src. 
			.4 CMakeLists.txt. 
			.4 package.xml. 
		.3 file :  \begin{minipage}[t]{10cm}
						File Package
						{.}
				\end{minipage}.
			.4 include. 
			.4 src. 
				.5 \textcolor{margincolor}{file.cpp} : 
						\begin{minipage}[t]{10cm}
						File generated with \faust compiler
						{.}
					   \end{minipage}. 
			.4 CMakeLists.txt. 
			.4 package.xml. 
}

%\section{Renaming DSP file}
%If the dsp file name does not fit you, you can change it using the \lstinline'-o' command.
%For instance, if you want the package generated from DSP file to have a different name that your DSP file name, you can type :
%\begin{lstlisting}
%	faust2ros -o foobar file.dsp
%\end{lstlisting}
%The output is going to be : \\
%
%\dirtree{%
%.1 foobar.zip. 
%	.2 faust\_msgs. 
%	.2 foobar. 
%}
%\newpage
\section{Example}
Here is an example of three files compilation.\\

Input :
\begin{lstlisting}
faust2rosgtk -install foo_ws -o foo1 file1.dsp 
			 -install foo_ws -o foo2 file2.dsp 
			 -install bar_ws -o bar file3.dsp
\end{lstlisting}
\newpage
Output :\\
\dirtree{%
.1 \~{ }. 	
	.2 foo\_ws. 
		.3 faust\_msgs. 
		.3 foo1. 
		.3 foo2. 
	.2 bar\_ws. 
		.3 bar. 
}